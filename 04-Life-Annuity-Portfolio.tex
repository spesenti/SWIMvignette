\documentclass[]{article}
\usepackage{lmodern}
\usepackage{amssymb,amsmath}
\usepackage{ifxetex,ifluatex}
\usepackage{fixltx2e} % provides \textsubscript
\ifnum 0\ifxetex 1\fi\ifluatex 1\fi=0 % if pdftex
  \usepackage[T1]{fontenc}
  \usepackage[utf8]{inputenc}
\else % if luatex or xelatex
  \ifxetex
    \usepackage{mathspec}
  \else
    \usepackage{fontspec}
  \fi
  \defaultfontfeatures{Ligatures=TeX,Scale=MatchLowercase}
\fi
% use upquote if available, for straight quotes in verbatim environments
\IfFileExists{upquote.sty}{\usepackage{upquote}}{}
% use microtype if available
\IfFileExists{microtype.sty}{%
\usepackage{microtype}
\UseMicrotypeSet[protrusion]{basicmath} % disable protrusion for tt fonts
}{}
\usepackage[margin=1in]{geometry}
\usepackage{hyperref}
\hypersetup{unicode=true,
            pdfborder={0 0 0},
            breaklinks=true}
\urlstyle{same}  % don't use monospace font for urls
\usepackage{graphicx,grffile}
\makeatletter
\def\maxwidth{\ifdim\Gin@nat@width>\linewidth\linewidth\else\Gin@nat@width\fi}
\def\maxheight{\ifdim\Gin@nat@height>\textheight\textheight\else\Gin@nat@height\fi}
\makeatother
% Scale images if necessary, so that they will not overflow the page
% margins by default, and it is still possible to overwrite the defaults
% using explicit options in \includegraphics[width, height, ...]{}
\setkeys{Gin}{width=\maxwidth,height=\maxheight,keepaspectratio}
\IfFileExists{parskip.sty}{%
\usepackage{parskip}
}{% else
\setlength{\parindent}{0pt}
\setlength{\parskip}{6pt plus 2pt minus 1pt}
}
\setlength{\emergencystretch}{3em}  % prevent overfull lines
\providecommand{\tightlist}{%
  \setlength{\itemsep}{0pt}\setlength{\parskip}{0pt}}
\setcounter{secnumdepth}{0}
% Redefines (sub)paragraphs to behave more like sections
\ifx\paragraph\undefined\else
\let\oldparagraph\paragraph
\renewcommand{\paragraph}[1]{\oldparagraph{#1}\mbox{}}
\fi
\ifx\subparagraph\undefined\else
\let\oldsubparagraph\subparagraph
\renewcommand{\subparagraph}[1]{\oldsubparagraph{#1}\mbox{}}
\fi

%%% Use protect on footnotes to avoid problems with footnotes in titles
\let\rmarkdownfootnote\footnote%
\def\footnote{\protect\rmarkdownfootnote}

%%% Change title format to be more compact
\usepackage{titling}

% Create subtitle command for use in maketitle
\providecommand{\subtitle}[1]{
  \posttitle{
    \begin{center}\large#1\end{center}
    }
}

\setlength{\droptitle}{-2em}

  \title{}
    \pretitle{\vspace{\droptitle}}
  \posttitle{}
    \author{}
    \preauthor{}\postauthor{}
    \date{}
    \predate{}\postdate{}
  

\begin{document}

\hypertarget{life-annuity-portfolio}{%
\section{Life Annuity Portfolio}\label{life-annuity-portfolio}}

In this section we showcase the SWIM package to conduct sensitivity
analysis of a life annuity portfolio.

Model inspired by Olivieri \& Pitacco (2003). See also Hari et
al.~(2009) for an application to a pension scheme. Point of view of an
insurer who aims at building up a partial internal model for a portfolio
of life annuities.

Types of risks:\\
+ The \emph{mortality risk} is originated by the random lifetimes of
annuitants. It can be split in process and longevity risk.\\
- The \emph{investment risk} originates from the randomness stemming
from financial markets in which the insurer invests.

\textbf{Assumptions}\textbackslash{} Annuitants same age \(x\) at entry
time \(t=0\) (when they buy the annuity) \(\rightarrow\) single cohort.
In \(t=n\) every policy holder is deceased. Same immediate life annuity
(\(R\) annual amount) for every policy holder.
\textbackslash{}\textbackslash{} Notation:

\begin{itemize}
    \item $Z_t$ portfolio fund at time t ;
    \item $N_t$  number of annuitants at time t;
    \item $D_t = N_t - N_{t+1}$;
    \item $I_t$ rate of return for the period ($t-1,t$);
    \item $\omega -1 $ maximum attainable age;
    \item $\Pi$ single premium;
    \item $M$ (initial) solvency margin.
\end{itemize}

At time \(t=0\), initial fund: \begin{displaymath}
Z_0 = N_0 \Pi + M.
\end{displaymath} Portfolio fund dynamics: \begin{displaymath}
Z_t = (Z_{t-1} - N_{t-1} R) (1+I_{t}), \ \ \ t=1,\ldots,\omega-x. 
\end{displaymath} For a fixed solvency level \(\varepsilon\), the
initial margin \(M\) is chosen so that: \begin{displaymath}
P(Z_n \ge 0) \ge 1-\varepsilon.
\end{displaymath} \newpage

\subsection{Mortality Model}

The model considers simultaneously age \(x\) and calendar year
\(t\).\textbackslash{}\textbackslash{} The random variable
\(\tilde{q}_{x+t,t}\) gives the one-year death probability for an
individual aged x+t in calendar year \(\hat{t}+t\) (where \(\hat{t}\) is
a fixed year). \textbackslash{}\textbackslash{} Under the assumptions of
homogeneous and indipendent lives, the conditional probability
distribution of \(D_t\) is Binomial: \begin{equation}
D_t|N_{t},\tilde{q}_{x+t,t} \sim \text{Binom}(N_{t},\tilde{q}_{x+t,t}), \ \ t=0,\ldots,\omega-x-1. \label{binom}
\end{equation} In \eqref{binom}, the uncertainty of
\(\tilde{q}_{x+t,t}\) represents the longevity risk.
\textbackslash{}\textbackslash{}
\textbf{Longevity Risk}\textbackslash{}\textbackslash{} Model: M7
(Quadratic CBD model with cohort effects) \(\rightarrow\) extension of
Lee-Carter model. See Cairns et
al.~(2009).\textbackslash{}\textbackslash{} Be careful: notation here is
a bit different! (the single cohort of policy holders in the life
annuity portfolio allows a simplified notation):

\begin{itemize}
    \item $D_{x,t}$ number of deaths in a reference population at age $x$ last birthday during calendar year $t$;
    \item $E_{x,t}^0$ initial exposed to risk  at age $x$ in year $t$;
    \item $\tilde{q}_{x,t}$ one-year death probability for an individual aged $x$ last birthday and in calendar year $t$.
    \end{itemize}

Structure of M7. See GAPC family \(\rightarrow\) StMoMo (Villegas et
al.~(2018): \begin{displaymath}
\begin{split}
& (1) \ D_{x,t} \sim \text{Binom}(E_{x,t}^0 \ , \ \tilde{q}_{x,t}), \\
& (2) \ \text{logit} (\tilde{q}_{x,t}) = \eta_{x,t}, \\
& (3) \ \eta_{x,t} = \kappa_t^{(1)} + (x-\overline{x}) \kappa_t^{(2)} + ((x-\overline{x}) - \tilde{\sigma}_x^2)^2  \kappa_t^{(3)} + \gamma_{t-x}.
\end{split}
\end{displaymath} Model M7 is fitted to Italy male mortality data for
ages from 55 to 89 and calendar years from 1950 to 2014 \(\rightarrow\)
M7 is the best.

\subsection{Financial Risk}

\(I_t\) is defined by the equation: \begin{displaymath}
1+I_t = \frac{S_{t+1}}{S_t}.
\end{displaymath} \(S_t\) is a Geometric Brownian motion:
\begin{equation}
\frac{dS_t}{S_t} = \mu d_t + \sigma dW_t, \label{sde}
\end{equation} where \(W_{t}\) is a Wiener process, \(\mu\) (percentage
drift) and \(\sigma\) (percentage volatility) are constants. The
stochastic differential equation \eqref{sde} admits analytic solution:
\begin{displaymath}
 S_{t}=S_{0}\exp \left(\left(\mu -{\frac {\sigma ^{2}}{2}}\right)t+\sigma W_{t}\right),
\end{displaymath} hence \begin{displaymath}
1+I_t = \text{exp}\left( \left(\mu-\frac{\sigma^2}{2}\right)  +\sigma(W_{t+1}-W_t)  \right).
\end{displaymath}

\subsection{Reverse sensitivity analysis}

\textbf{Output} variable: \begin{displaymath}
Y = -Z_n.
\end{displaymath} The opposite of the fund at time \(n\) is chosen as
output variable to allow adverse scenarios as stress.
\textbackslash{}\textbackslash{} Risk Factors \(\textbf{X}\):

\begin{table}[h]
    \centering
    \begin{tabular}{@{}cccc@{}}
        \toprule
        & Mortality  & Yield  & Risk Factors        \\ \midrule
        Model 4  & Full Stochastic & Stochastic  & $N_t \ \ \tilde{q}_{x+t,t} \ \ I_t$   \\ \bottomrule
    \end{tabular}
\end{table}

\(\\\) We have: \begin{displaymath}
Y = g((N_t), (I_t), (\tilde{q}_{x+t,t}))
\end{displaymath} where \(g\) is implicitly defined by the equations in
previous sections.\textbackslash{}\textbackslash{} Stress:
\begin{displaymath}
\begin{split}
\alpha &=0.9, \ \ \ \ \ q=VaR_\alpha^Q(Y)=0, \\ 
stress_{ES}&=0.2, \ \ \ \ \  s=ES_\alpha^Q(Y)=\E(Y|Y>q) \ (1+stress_{ES}).
\end{split}
\end{displaymath}


\end{document}
